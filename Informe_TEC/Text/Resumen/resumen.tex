Se presenta un proyecto en el cual nuestra empresa ha sido encargada de implementar una infraestructura de red 4G con el propósito de brindar servicios a vehículos conectados en una autopista específica en Castilla y León. El tramo de carretera seleccionado abarca 73 kilómetros, donde se aprovechará la infraestructura de red 4G existente y se instalarán equipos adicionales en las torres de telefonía ya disponibles a lo largo de la carretera. El objetivo principal de este proyecto es ofrecer un servicio ininterrumpido con un ancho de banda suficiente para transmitir vídeo captado por cámaras en los vehículos, lo cual permitirá a los usuarios detectar señales de tráfico, mejorar la seguridad vial, cumplir con las regulaciones de tráfico, así como optimizar el consumo de combustible y reducir las emisiones contaminantes. Para lograr estos objetivos, se propone el uso de técnicas avanzadas de procesamiento de imágenes y aprendizaje automático, como el modelo YOLO, con el fin de lograr una detección precisa y eficiente de las señales de tráfico, contribuyendo así a la seguridad vial y ofreciendo soluciones innovadoras.\\

Se han considerado varias decisiones y aspectos relacionados con la infraestructura de red. Se ha optado por alquilar torres existentes y utilizar componentes propios para diseñar una red que cumpla con los requisitos económicos y proporcione cierto grado de control en caso de problemas. Se han explorado opciones como la construcción de torres propias, el funcionamiento como operador virtual o la adquisición e instalación de equipos propios. Se destaca la selección de una estación base de la empresa china Baicells y una antena de la empresa irlandesa Alpha Wireless, junto con sus características técnicas relevantes. Además, se menciona el uso de la herramienta Xirio-online para simular la infraestructura y realizar la planificación y el diseño de las redes de telecomunicaciones. Asimismo, se plantea la opción de alquilar servicios de AWS en lugar de adquirir una gran cantidad de servidores para el procesamiento de datos. Se enfatiza la importancia de establecer una conexión dedicada de alta velocidad y confiabilidad para gestionar eficientemente el tráfico de datos en la red 4G.\\

Para la detección de objetos, se utiliza el algoritmo YOLO (You Only Look Once), el cual permite una detección rápida y precisa en una imagen completa. Para evaluar el rendimiento de la detección, se utilizan métricas como la precisión y la recuperación.\\

Además, se ha explicado el proceso burocrático necesario para enviar la solicitud y realizar el pago de las tasas correspondientes mediante los modelos 790 y 990, requisitos esenciales para llevar a cabo el proyecto. También se ha elaborado una tabla con los costos y amortizaciones asociados al proyecto.\\

Por último, se ha realizado un análisis del rendimiento de la red utilizando la versión estándar de Linux y aplicando un kernel de baja latencia.