Como en la tecnología LTE cada canal es de 15MHz, nos encontramos en una situación de banda estrecha. Ante estas situaciones podemos dividir el procedimiento en tres etapas:
\begin{itemize}
\item Pago tasas del modelo 790.
\item Envío de la solicitud.
\item Pago tasas del modelo 990.
\end{itemize}

\subsection{Pago tasas del modelo 790}

\begin{quote}
\itshape
De acuerdo con la ley 39/2015, de 1 de octubre, del Procedimiento Administrativo Común de las Administraciones Públicas, y el Real Decreto 123/2017, de 24 de febrero, por el que se aprueba el Reglamento sobre el uso del dominio público radioeléctrico, la tramitación de los procedimientos relativos al espectro radioeléctrico, así como la relación con los órganos competentes del Ministerio a este respecto, se deberá llevar a cabo obligatoriamente por medios electrónicos, siempre que estén disponibles en la sede electrónica del Ministerio.\\
Se puede acceder a los procedimientos relacionados con el Servicio Móvil y Fijo de Banda Estrecha de la Subdirección General de Planificación y Gestión del Espectro Radioeléctrico en la sede electrónica del Ministerio.\\

El Real Decreto 1620/2005, de 30 de diciembre, por el que se regulan las tasas establecidas en la Ley 32/2003, de 3 de noviembre, General de Telecomunicaciones, completa y desarrolla la regulación de la referida Ley General de Telecomunicaciones, precisando las reglas y criterios aplicables para la fijación de las tasas y estableciendo el procedimiento para su liquidación. Esta resolución tiene por objeto establecer el procedimiento para la autoliquidación y las condiciones para el pago por vía telemática de las tasas de telecomunicaciones establecidas en el anexo I, apartado 4 de la Ley 32/2003, de 3 de noviembre, General de Telecomunicaciones:
\end{quote}


El artículo 4 anteriormente mencionado: Tasas de telecomunicaciones presenta que,\\
\begin{quote}
\itshape
La gestión precisa para la emisión de certificaciones registrales y de la presentación de proyecto técnico y del certificado o boletín de instalación que ampara las infraestructuras comunes de telecomunicaciones en el interior de edificios, de cumplimiento de las especificaciones técnicas de equipos y aparatos de telecomunicaciones, así como la emisión de dictámenes técnicos de evaluación de la conformidad de estos equipos y aparatos, las inscripciones en el registro de instaladores de telecomunicación, las actuaciones inspectoras o de comprobación técnica que, con carácter obligatorio, vengan establecidas en esta ley o en otras disposiciones con rango legal, la tramitación de autorizaciones o concesiones demaniales para el uso privativo del dominio público radioeléctrico y la tramitación de autorizaciones de uso especial de dicho dominio darán derecho a la exacción de las tasas compensatorias del coste de los trámites y actuaciones necesarias, con arreglo a lo que se dispone en los párrafos siguientes.\\

La cuantía de la tasa se establecerá en la Ley de Presupuestos Generales del Estado. La tasa se devengará en el momento de la solicitud correspondiente. El rendimiento de la tasa se ingresará en el Tesoro Público o, en su caso, en las cuentas bancarias habilitadas al efecto respectivamente por la Comisión del Mercado de las Telecomunicaciones o por la Agencia Estatal de Radiocomunicaciones en los términos previstos en los artículos 47 y 48 de esta ley, en la forma que reglamentariamente se determine. Asimismo, reglamentariamente se establecerá la forma de liquidación de la tasa.
\end{quote}

Por lo que se realiza el pago de la tasa 790, que se puede encontrar en la sede electrónica en el apartado de “\textbf{Pago de tasas de tramitación de Telecomunicaciones. Modelo 790}”. En “\textbf{Redes Radioeléctricas del Servicio Móvil y Fijo de Banda Estrecha}”.\\

Una vez realizado el pago de la tasa 790, se obtiene un justificante de dicho pago que se adjuntará con el paso 2, explicado a continuación.

\subsection{Envío de la solicitud}


El Reglamento de uso del dominio público radioeléctrico aprobado por el Real Decreto 123/2017, de 24 de febrero, considera usos experimentales los destinados a efectuar pruebas técnicas o ensayos sobre propagación, utilización de nuevas bandas de frecuencia o demostraciones de nuevos servicios o tecnologías. Su duración está limitada a máximo seis meses. 
La solicitud de título habilitante (autorización individual) para el uso del dominio público radioeléctrico para la cobertura de eventos de corta duración se debe presentar ante esta Secretaría de Estado con una antelación de, al menos, diez días hábiles al comienzo del período de utilización solicitado.\\ 

Junto con la solicitud, se debe presentar una memoria técnica (fichero estructurado XML) que incluirá el período de utilización, una descripción del servicio a prestar, la red radioeléctrica, con las estaciones y equipos que se pretenden utilizar, con indicación de sus características técnicas y ubicación. Dicha memoria técnica puede estar firmada por un técnico competente o en otro caso venir firmada electrónicamente por el titular responsable de la red o su representante legal. 

Para generar la descripción de la red, se facilita la herramienta SM\_GenXML, se completa el fichero descriptivo de la red y se realiza el proceso de firma del mismo. \\

La solicitud de título habilitante de uso de frecuencias para eventos de corta duración o para pruebas experimentales debe presentarse utilizando el procedimiento Solicitud Nuevas Redes Radioeléctricas Servicio Móvil y Fijo de banda estrecha correspondiente al conjunto de procedimientos relativos a “Redes Radioeléctricas del Servicio Móvil y Fijo de banda estrecha” disponibles en la sede electrónica:
Respecto al punto “Anexado de documentos” del Manual indicado, en lo relativo al documento “Datos Adicionales No Estructurados del Proyecto” deberán facilitarse las fechas del evento, frecuencias alternativas a las grabadas en el fichero, rango de funcionamiento en frecuencias de los equipos y cualquier otra información necesaria o que pueda facilitar el estudio de la red por parte de la Administración 
Durante el proceso de generación de la descripción de red mediante la herramienta SM\_GenXML se deberá adjuntar este fichero firmado electrónicamente .

\subsection{Pago tasas del modelo 990}
En el anexo I, apartado 3 de la Ley 32/2003, de 3 de noviembre, General de Telecomunicaciones:\\

El artículo 3 en el que se describe la tasa por reserva del dominio público radioeléctrico presenta que,
\begin{quote}
\itshape
La reserva para uso privativo de cualquier frecuencia del dominio público radioeléctrico a favor de una o varias personas o entidades se gravará con una tasa anual, en los términos que se establecen en este apartado.\\
Para la fijación del importe a satisfacer en concepto de esta tasa por los sujetos obligados, se tendrá en cuenta el valor de mercado del uso de la frecuencia reservada y la rentabilidad que de él pudiera obtener el beneficiario.\\
Para la determinación del citado valor de mercado y de la posible rentabilidad obtenida por el beneficiario de la reserva se tomarán en consideración, entre otros, los siguientes parámetros:\\
a) El grado de utilización y congestión de las distintas bandas y en las distintas zonas geográficas.\\
b) El tipo de servicio para el que se pretende utilizar la reserva y, en particular, si éste lleva aparejadas las obligaciones de servicio público recogidas en el título III.\\
c) La banda o sub-banda del espectro que se reserve.\\
d) Los equipos y tecnología que se empleen.\\
e) El valor económico derivado del uso o aprovechamiento del dominio público reservado.
\end{quote}
En el artículo 85, apartado 3 de la Ley 31/2022, de 23 de diciembre, de Presupuestos Generales del Estado para el año 2023 se especifica el importe correspondiente al pago de la tasa 990, \\
\begin{quote}
\itshape
La tasa por reserva de dominio público radioeléctrico establecida en el apartado 3 del anexo I de la Ley 11/2022, de 28 de junio, General de Telecomunicaciones (en adelante Ley General de Telecomunicaciones), ha de calcularse mediante la expresión:

$$T = \frac{[N \times V]}{166,386} = \frac{[S \, (\text{km}^2) \times B \, (\text{kHz}) \times F \, (C1, C2, C3, C4, C5)]}{166,386}$$

Donde:\\
T = importe de la tasa anual en euros.\\
N = número de unidades de reserva radioeléctrica (URR) calculado como el producto de S x B, es decir, superficie en kilómetros cuadrados de la zona de servicio, por ancho de banda reservado expresado en KHz.\\
V = valor de la URR, determinado en función de los cinco coeficientes Ci, establecidos en la Ley General de Telecomunicaciones, y cuya cuantificación, de conformidad con dicha Ley, será establecida en la Ley de Presupuestos Generales del Estado.\\
F (C1, C2, C3, C4, C5) = esta función es el producto de los cinco coeficientes indicados anteriormente.
\end{quote}

En el presente proyecto, $N\ =\ 10 \times 10^3$\, S= 10$Km^2$, los valores de los coeficientes corresponden a los descritos dentro del escenario del apartado 1.1.2: Servicio móvil asignación fija/frecuencia compartida/zona de alta utilización (ya que la red se encuentra en un área metropolitana)/autoprestación. Y por tanto: 
\begin{itemize}
\item C1=1,25
\item C2=1,25
\item C3=1,1
\item C4=1,3
\item C5=0,4590
\end{itemize}

Una vez realizadas las cuentas, se obtiene un valor de importe aproximado de T = 616,3848 \euro{} anuales. Como es una red experimental, su uso está limitado a 6 meses, por lo que el pago de la tasa 990 es de 308,1924. \\

En los Anexos se adjuntan...
