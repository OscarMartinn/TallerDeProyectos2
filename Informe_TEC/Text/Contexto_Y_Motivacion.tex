La empresa para la que trabajamos ha sido adjudicataria de un contrato con el objetivo de desarrollar un servicio de vehículo conectado consistente en el envío de información de video desde el vehículo a un servidor en el cloud. El video será procesado en el cloud para la detección automática, mediante técnicas de inteligencia artificial, de señales de tráfico en la carretera.\\

El objetivo de este proyecto es desarrollar una infraestructura de red 4G que permita la transmisión de video desde los vehículos conectados al servidor en el cloud. Esta infraestructura permitirá a la empresa ofrecer un servicio de predicción de señales de tráfico en la carretera, lo que ayudará a mejorar la seguridad y la eficiencia en el tráfico. Esto se logrará desarrollando un demostrador previo al despliegue de la red para probar la capacidad de la empresa para desplegar el servicio, y presentando una memoria técnicoeconómica que detalle el despliegue de la red en el tramo de autovía A-62 desde el kilómetro 158 hasta el kilómetro 231. Además, la empresa tendrá que demostrar contar con los permisos legales pertinentes para el despliegue del servicio, que tendrá una duración máxima de 6 meses.