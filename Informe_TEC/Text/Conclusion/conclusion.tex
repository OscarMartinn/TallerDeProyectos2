En conclusión, nuestro proyecto ha logrado establecer una conexión robusta y fiable mediante una red 4G, permitiendo una comunicación fluida entre el coche AWS DeepRacer y el sistema de reconocimiento de señales de tráfico basado en IA. La combinación de un rendimiento óptimo de la red, el uso de YOLO como algoritmo de IA y la capacidad de respuesta instantánea del coche ha resultado en un proyecto exitoso y prometedor.\\

La red 4G ha brindado una comunicación estable y confiable, mientras que el rendimiento de la red, incluyendo el ancho de banda, el jitter y la latencia mínima, ha permitido un streaming fluido y una respuesta instantánea del coche a las señales reconocidas. La elección de YOLO como algoritmo de IA ha sido acertada debido a su rendimiento sobresaliente en términos de velocidad y precisión en tiempo real. En términos de velocidad, precisión y capacidad de detección en tiempo real. Su enfoque basado en grid y su amplia adopción en la comunidad de IA respaldan su idoneidad para nuestra aplicación y nos brindan resultados confiables y efectivos en la detección de señales de tráfico. En conjunto, este proyecto destaca el potencial de la combinación de tecnologías avanzadas para mejorar la seguridad vial y la interacción entre sistemas inteligentes y el entorno físico.\\

Al analizar el tráfico promedio en la autovía, que es aproximadamente de 740 coches, y considerando que nuestra estación puede atender como máximo a 280 usuarios simultáneamente, se plantea la necesidad de explorar soluciones para mejorar la capacidad y el rendimiento de la red.\\

Otra consideración importante es el uso de edge computing. Actualmente, el procesamiento se realiza en la nube, pero es posible que trasladarlo al borde de la red (edge) pueda ofrecer mejores resultados. El procesamiento dentro del vehículo permite una respuesta más rápida y eficiente, ya que los datos se pueden analizar y actuar localmente sin tener que transmitirlos a servidores remotos. Al alquilar servidores, puedes aprovechar la infraestructura externa para realizar tareas que requieran mayor capacidad de procesamiento o almacenamiento, mientras que el procesamiento básico se puede realizar en el vehículo mismo.\\

En relación al alquiler de servidores, se ha tomado la decisión debido a la tendencia creciente de utilizar el procesamiento de datos dentro del vehículo. Esto significa que en el futuro es probable que los vehículos cuenten con mayor capacidad de procesamiento interno, lo que reduce la dependencia de servidores externos y permite un procesamiento más rápido y eficiente de los datos en el propio vehículo.