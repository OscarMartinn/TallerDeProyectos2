
La empresa para la que trabajamos ha sido adjudicataria de un contrato con el objetivo de desarrollar un servicio de vehículo conectado consistente en el envío de información de vídeo desde el vehículo a un servidor en la nube. El vídeo será procesado en \textit{Cloud} para la detección automática, mediante técnicas de inteligencia artificial, de señales de tráfico en la carretera.\\

El objetivo de este proyecto es desarrollar una infraestructura de red 4G que permita la transmisión de vídeo desde los vehículos conectados al servidor \textit{Cloud}. Esta infraestructura permitirá a la empresa ofrecer un servicio de predicción de señales de tráfico en la carretera, lo que ayudará a mejorar la seguridad y la eficiencia en el tráfico. Esto se logrará desarrollando un demostrador previo al despliegue de la red para probar la capacidad de la empresa para desplegar el servicio, y presentando una memoria técnico-económica que detalle el despliegue de la red en el tramo de autovía A-62 desde el kilómetro 158 hasta el kilómetro 231. Además, la empresa tendrá que demostrar que cuenta con los permisos legales pertinentes para el despliegue del servicio, que tendrá una duración máxima de 6 meses.

Los componentes que la empresa pone a nuestra disposición son herramientas esenciales para el desarrollo de proyectos en infraestructura de telecomunicaciones y en inteligencia artificial.\\

En primer lugar, contamos con ordenadores de sobremesa y memorias USB, que nos permiten trabajar en el desarrollo de algoritmos de inteligencia artificial, en la programación de los módems Huawei y las tarjetas SIM de Vodafone S.A.U. para la emulación de la red 4G.\\

Asimismo, disponemos de un sistema SDR (\textit{Software Defined Radio}), en concreto una BladeRF 2.0 xA9, un dispositivo que permite la recepción y transmisión de señales de radio en un amplio rango de frecuencias. Este sistema se controla mediante el \textit{software} abierto srsRAN, que nos permite configurar y gestionar las comunicaciones.\\

En cuanto a las frecuencias, se usan las de banda 7. En concreto el rango que va de 2540 a 2550 MHz y de 2650 a 2670 MHz. Estas frecuencias se utilizan para probar y validar el funcionamiento de las comunicaciones.\\

Para emular la transmisión de vídeo, la empresa nos proporciona vehículos Amazon DeepRacer Evo dirigidos por control remoto, junto con la información básica de su manejo. Estos vehículos nos permiten simular las condiciones reales de transmisión de vídeo en diferentes entornos.\\

Además, contamos con un conjunto modular de pistas y señales que nos permiten montar diversos circuitos para probar y validar diferentes escenarios de comunicación.\\

Por último, la empresa nos proporciona bibliotecas Python para aprendizaje automático y recursos de computación en la nube para ejecutar los algoritmos de aprendizaje automático. Estas herramientas nos permiten desarrollar y probar modelos de inteligencia artificial para mejorar la eficiencia y la seguridad de las comunicaciones.